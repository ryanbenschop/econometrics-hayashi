\documentclass{article}
\usepackage{amsmath}
\usepackage{amsthm}
\usepackage[english]{babel}
\usepackage[utf8]{inputenc}
\usepackage{graphicx}
\usepackage[textwidth=15cm]{geometry}
\usepackage{amssymb}
\usepackage[shortlabels]{enumitem}
\usepackage[dvipsnames]{xcolor}
\newtheorem{theorem}{Theorem}[section]
\newtheorem{corollary}{Corollary}[theorem]
\newtheorem{lemma}[theorem]{Lemma}
\theoremstyle{definition}
\newtheorem{definition}{Definition}[section]
\setlength{\parindent}{0pt}
\DeclareMathOperator*{\plim}{plim}
\DeclareMathOperator*{\Avar}{Avar}
\DeclareMathOperator*{\Corr}{Corr}
\DeclareMathOperator*{\rank}{rank}
\DeclareMathOperator*{\2sls}{2SLS}
\DeclareMathOperator*{\iv}{IV}
\DeclareMathOperator*{\gmm}{GMM}
\newcommand{\definitionbox}[1]{\fcolorbox{black}{Dandelion}{\begin{minipage}{14.5cm}{#1}\end{minipage}}}
\newcommand{\theorembox}[1]{\fcolorbox{black}{CornflowerBlue}{\begin{minipage}{14.5cm}{#1}\end{minipage}}}
\newcommand{\claim}[1]{
    \hrulefill{}
    \medbreak{}
    \textit{Claim: #1}
    
    \hrulefill{}
    \medbreak{}
}

\title{Large Sample Theory – Analytical Exercises}
\author{Ryan Benschop}
\date{}

\begin{document}
\maketitle

\textbf{1.} \claim{
    Define the stochastic process ${\{z_n\}}_{n \in \mathbb N}$ by:
    \begin{align*}
        & \mathbb P(z_n = 0) = 1 - n^{-1} \\
        & \mathbb P(z_n = n^2) = n^{-1}
    \end{align*}
    Then, $z_n \xrightarrow{p} 0$ and $\mathbb E z_n \to \infty$.
}

\begin{proof}
    To show that $z_n \xrightarrow{p} 0$, we need to show that for any $\varepsilon, \delta \in \mathbb R_{++}$, there is $N \in \mathbb N$ such that, if
    $n \ge N$, $\mathbb P(|z_n| > \varepsilon) < \delta$.

    \medbreak{}

    Choose any $\varepsilon, \delta \in \mathbb R_{++}$. Let $N \in \mathbb N$ be such that $N > \delta^{-1}$. Notice that the event $|z_n| > \varepsilon$ is
    a subset of the complementary event of $z_n = 0$. Hence:
    \begin{align*}
        \mathbb P(|z_n| > \varepsilon) &\le 1 - \mathbb P(z_n = 0) \\
        &= 1 - (1 - n^{-1}) \\
        &= n^{-1} \\
        &< \delta
    \end{align*}

    Next, to show that $\mathbb E z_n \to \infty$, we need to show that, for any $\Delta \in \mathbb R$, there is $N \in \mathbb N$ such that, if $n \ge N$,
    $\mathbb E z_n > \Delta$.

    \medbreak{}

    Choose any $\Delta \in \mathbb R$. Let $N \in \mathbb N$ be such that $N > \Delta$. Notice, $\mathbb E(z_n) = n$. Hence, if $n \ge N$,
    $\mathbb E(z_n) > \Delta$.
\end{proof}

\newpage

\textbf{2. (Chebychev's weak law of large numbers.)} \claim{
    Let ${\{z_n\}}_{n \in \mathbb N}$ be a stochastic process. Suppose $\mathbb E \bar z_n = \mu$ and $\mathbb V z_n = 0$. Then, $\bar z_n \xrightarrow{p} \mu$.
}

\begin{proof}
    To prove the theorem, we can show that under the given assumptions, $\bar z_n \xrightarrow{m.s.} \mu$, noting that convergence in mean square imples convergence
    in probability. We need to show that $\mathbb E[{(z_n - \mu)}^2] \to 0$.

    \medbreak{}

    Note:
    \begin{align*}
        \mathbb E[{(z_n - \mu)}^2] &= \mathbb E[{(\bar z_n - \mathbb E[\bar z_n] + \mathbb E[\bar z_n] - \mu)}^2] \\
        &= \mathbb E[{(\bar z_n - \mathbb E[\bar z_n])}^2 - 2(\bar z_n - \mathbb E[\bar z_n])(\mathbb E[\bar z_n] - \mu) + {(\mathbb E[\bar z_n] - \mu)}^2]
    \end{align*}

    Next, note:
    \begin{align*}
        (\bar z_n - \mathbb E[\bar z_n])(\mathbb E[\bar z_n] - \mu) &= \mathbb E[\bar z_n \mathbb E[\bar z_n] - \bar z_n \mu - {(\mathbb E [\bar z_n])}^2 +
        \mathbb E[\bar z_n] \mu] \\
        &= {(\mathbb E[\bar z_n])}^2 - \mathbb E[\bar z_n] \mu - {(\mathbb E[\bar z_n])}^2 + \mathbb E[\bar z_n] \mu \\
        &= 0
    \end{align*}

    So:
    \begin{align*}
        \mathbb E[{(z_n - \mu)}^2] &= \mathbb E[{(\bar z_n - \mathbb E[\bar z_n])}^2 + {(\mathbb E[\bar z_n] - \mu)}^2] \\
        &= \mathbb E[{(\bar z_n - \mathbb E[\bar z_n])}^2] + \mathbb E[{(\mathbb E[\bar z_n] - \mu)}^2] \\
        &= \mathbb V[\bar z_n] + {(\mathbb E[\bar z_n])}^2 - 2 \mathbb E[\bar z_n] \mu + \mu^2 \\
        &\to 0 + \mu^2 - 2\mu^2 + \mu^2 = 0
    \end{align*}
\end{proof}

\newpage

\textbf{3. (Consistency and asymptotic normality of OLS for random samples)} \claim{
    Consider the stochastic process ${\{(y_i, x_i)\}}_{i \in \mathbb{N}}$ where, for each $i \in \mathbb{N}$, $y_i$ is $\mathbb R$-valued, and $x_i$ is
    $\mathbb R^K$-valued. Suppose the following assumptions hold:
    \begin{itemize}
        \item (Linearity): there is $\beta \in \mathbb R^K$ such that, for each $i \in \mathbb N$, $y_i = x_i'\beta + \varepsilon_i$.
        \item (Random sampling): ${\{(y_i, x_i)\}}_{i \in \mathbb{N}}$ is an i.i.d.\ process.
        \item (Predetermined regressors): for each $i \in \mathbb{N}$, $\mathbb E[x_i \varepsilon_i] = 0$.
        \item (Rank condition): $\mathbb E[x_i x_i']$ exists and is nonsingular.
    \end{itemize}
    Let $\hat \beta$ denote the OLS estimator of $\beta$. Then, $\hat \beta$ is consistent.
    Suppose, further, that $\mathbb E[\varepsilon_i^2 x_i x_i']$ exists and is finite. Then, the asymptotic distribution related to $\hat \beta$ is given by:
    \[n^{\frac 12} (\hat \beta - \beta) \xrightarrow{d} \mathcal N(0, \text{Asy.} \mathbb V[\hat \beta])\]
    where:
    \[\text{Asy.} \mathbb V[\hat \beta] = {\left( \mathbb E[x_i x_i'] \right)}^{-1} \mathbb E[\varepsilon_i^2 x_i x_i'] {\left( \mathbb E[x_i x_i'] \right)}^{-1}\]
}

\begin{proof}
    (Direct proof) Letting $X$ denote the data matrix, and rewriting the expression for the OLS estimator, we have:
    \begin{align*}
        \hat \beta &= {(X'X)}^{-1}X'y = {(X'X)}^{-1}X'(X\beta + \varepsilon) \\
        &= \beta + {(X'X)}^{-1}X'\varepsilon \\
        &= \beta + {\left( n^{-1} \sum_{i=1}^n x_i x_i' \right)}^{-1} n^{-1} \sum_{i=1}^n x_i \varepsilon_i
    \end{align*}

    Now, notice that, since ${\{(y_i, x_i)\}}_{i \in \mathbb{N}}$ is i.i.d., ${\{x_i x_i'\}}_{i \in \mathbb{N}}$ and
    ${\{x_i \varepsilon_i\}}_{i \in \mathbb{N}}$ are also i.i.d. (stochastic processes defined by measurable functions of i.i.d.\ processes are i.i.d.).
    Additionally, their means exist by assumption. Thus, by Kolmogorov's SLLN:\
    \begin{align*}
        n^{-1} \sum_{i=1}^n x_i x_i' \xrightarrow{a.s.} \mathbb E[x_i x_i'] && \text{and} && n^{-1} \sum_{i=1}^n x_i \varepsilon_i \xrightarrow{a.s.}
        \mathbb E[x_i \varepsilon_i]
    \end{align*}
    which implies that convergence in probability holds as well. Define $g: \mathbb R^{K, K} \times \mathbb R^{K, 1} \to \mathbb R^K$ by:
    \[g(A, B) = \beta + A^{-1} B\]
    Since we assumed $\mathbb E[x_i x_i']$ is invertible, noting that matrix inversion is a continuous transformation, the continuous mapping theorem implies:
    \begin{align*}
        \plim_{n \to \infty} \hat \beta &= \plim_{n \to \infty} g{\left(n^{-1} \sum_{i=1}^n x_i x_i', n^{-1} \sum_{i=1}^n x_i \varepsilon_i\right)}
        = g{\left(\plim_{n \to \infty}n^{-1} \sum_{i=1}^n x_i x_i', \plim_{n \to \infty} n^{-1} \sum_{i=1}^n x_i \varepsilon_i\right)} \\
        &= g(\mathbb E[x_i x_i'], \mathbb E[x_i \varepsilon_i]) = \beta + \mathbb E{[x_i x_i']}^{-1} \mathbb E[x_i \varepsilon_i]
    \end{align*}

    But, by assumption, $\mathbb E[x_i \varepsilon_i] = 0$. Hence, $\hat \beta \xrightarrow{p} \beta$.

    \medbreak{}

    Now, rearranging the above expression for $\hat \beta$, we have:
    \[n^{\frac 12} (\hat \beta - \beta) = {\left( n^{-1} \sum_{i=1}^n x_i x_i' \right)}^{-1} n^{-\frac 12} \sum_{i=1}^n x_i \varepsilon_i\]
    Recalling that ${\{x_i \varepsilon_i\}}_{i \in \mathbb{N}}$ is i.i.d., noting that its mean is zero and its variance exists by assumption, the Lindeberg-Levy
    central limit theorem implies:
    \[n^{-\frac 12} \sum_{i=1}^n x_i \varepsilon_i \xrightarrow{d} \mathcal N(0, \Omega)\]
    where $\Omega = \mathbb V[x_i \varepsilon_i] = \mathbb E[\varepsilon_i^2 x_i x_i']$.

    \medbreak{}

    Thus, by the continuous mapping theorem:
    \[n^{-\frac 12}(\hat \beta - \beta) \xrightarrow{d} {\left( \mathbb E[x_i x_i] \right)}^{-1} \Xi\]
    where $\Xi \sim \mathcal N(0, \Omega)$. Hence:
    \[n^{-\frac 12}(\hat \beta - \beta) \xrightarrow{d} \mathcal N(0, \text{Asy.} \mathbb V[\hat \beta])\]
    noting that ${(\mathbb E[x_i x_i'])}^{-1} \Omega {(\mathbb E[x_i x_i'])}^{-1} = {(\mathbb E[x_i x_i'])}^{-1} \mathbb E[\varepsilon_i^2 x_i x_i']
    {(\mathbb E[x_i x_i'])}^{-1} = \text{Asy.} \mathbb V[\hat \beta]$
\end{proof}

\begin{proof}
    (Using Proposition 2.1 in the text) Proposition 2.1.\ provides that under the assumptions of linearity, ergodic stationarity of
    ${\{(y_i, x_i)\}}_{i \in \mathbb{N}}$, predetermined regressors, and satisfaction of the rank condition, the OLS estimator is consistent. Moreover, under the
    additional assumption that ${\{x_i \varepsilon_i\}}_{i \in \mathbb{N}}$ is a martingale difference sequence with finite second moments, then:
    \[n^{\frac 12} (\hat \beta - \beta) \xrightarrow{d} \mathcal N(0, \text{Asy.} \mathbb V[\hat \beta ])\]

    Hence, we need only show that under the first set of assumptions in the present claim, ${\{(y_i, x_i)\}}_{i \in \mathbb{N}}$ is stationary ergodic and, under
    the further second moment assumption, ${\{x_i \varepsilon_i'\}}_{i \in \mathbb{N}}$ is a martingale difference sequence with finite second moments.

    \medbreak{}

    For $i \in \mathbb{N}$, let $z_i = (y_i, x_i)$. Let $i_1, \ldots, i_r$ be a set of indices, and fix any $i \in \mathbb{N}$ with $i \le i_1$, and
    $k \in \mathbb{N}$. Consider the joint distributions $(z_i, z_{i_1}, \ldots, z_{i_r})$ and $(z_{i+k}, z_{i_1+k}, \ldots, z_{i_r+k})$. Since $z_i$ is an
    independent process, each of these joint densities is simply the product of the marginal densities of the individual elements.\footnote{
        Note that `marginal density' here refers to the density of $z_i = (y_i, x_i)$.
    } Further, since the elements are identically distributed, each of these marginal densities is the same. Hence, the two joint densities are the same, so
    ${\{z_i\}}_{i \in \mathbb{N}}$ is a stationary process.

    \medbreak{}

    Next, let $f:\mathbb R^K \to \mathbb R$ and $\mathbb R^{\ell} \to \mathbb R$ (for any $\ell \in \mathbb{N}$) be any bounded functions. Since $z_i$ is
    an independent process:
    \[\mathbb E[f(z_i, \ldots, z_{i+k})g(z_{i+n}, \ldots, z_{i+n+\ell})] = \mathbb E[f(z_i, \ldots, z_{i+k})] \mathbb E[g(z_{i+n}, \ldots, z_{i+n+\ell})]\]
    for any $n \in \mathbb{N}$. Hence:
    \[\lim_{n \to \infty} {\left |\mathbb E[f(z_i, \ldots, z_{i+k})g(z_{i+n}, \ldots, z_{i+n+\ell})]\right|} = \mathbb E[f(z_i, \ldots, z_{i+k})] \mathbb E[g(z_{i+n}, \ldots, z_{i+n+\ell})]\]
    so that $z_i$ is ergodic.

    \medbreak{}

    Finally, to see that ${\{x_i \varepsilon_i\}}_{i \in \mathbb{N}}$ is a martingale difference sequence, notice that, since
    ${\{(y_i, x_i)\}}_{i \in \mathbb{N}}$ is i.i.d., ${\{x_i \varepsilon_i\}}_{i \in \mathbb{N}}$ is i.i.d.\ Thus,
    $\mathbb E[x_i \varepsilon_i | x_{i-1} \varepsilon_{i-1}, x_{i-2} \varepsilon_{i-2}, \ldots] = \mathbb E[x_i \varepsilon_i]$. But, by the assumption of
    predetermined regressors, this expectation is zero. Hence, ${\{x_i \varepsilon_i\}}_{i \in \mathbb{N}}$ is a martingale difference sequence.
\end{proof}

\newpage

\textbf{4. (Consistent estimation of $\mathbb E(\varepsilon_i^2 x_i x_i')$)} \claim{
    Consider the $\mathbb R^2$-valued stochastic process ${\{(y_i, x_i)\}}_{i \in \mathbb N}$. Suppose:
    \begin{itemize}
        \item For each $i \in \mathbb N$, $y_i = \beta x_i + \varepsilon_i$.
        \item ${\{(y_i, x_i)\}}$ is jointly stationary and ergodic.
        \item $\mathbb E[\varepsilon_i^2 x_i^2]$ exists and is finite.
        \item $\mathbb E[x_i^4]$ exists and is finite.
    \end{itemize}
    Let $\hat \beta$ be a consistent estimator of $\beta$. Then:
    \[n^{-1} \sum_{i=1}^n \hat \varepsilon_i^2 x_i^2 \xrightarrow{p} \mathbb E(\varepsilon_i^2 x_i^2)\]
}

\begin{proof}
    Note:
    \begin{align*}
        \hat \varepsilon_i &= y_i - \hat \beta x_i \\
        &= y_i - \hat \beta x_i + \beta x_i - \beta x_i \\
        &= y_i - \beta x_i - (\hat \beta - \beta) x_i \\
        &= \varepsilon_i - (\hat \beta - \beta) x_i \\
        \implies \hat \varepsilon_i^2 &= \varepsilon_i^2 - 2(\hat \beta - \beta) \varepsilon_i x_i + {(\hat \beta - \beta)}^2x_i^2 \\
        \implies \hat \varepsilon_i^2 x_i^2 &= \varepsilon_i^2 x_i^2 - 2(\hat \beta - \beta) \varepsilon_i x_i^3 + {(\hat \beta - \beta)}^2x_i^4
    \end{align*}

    Hence:
    \[n^{-1} \sum_{i=1}^n \hat \varepsilon_i^2 x_i^2 = n^{-1} \sum_{i=1}^n \varepsilon_i^2 x_i^2 - 2(\hat \beta - \beta)n^{-1} \sum_{i=1}^n \varepsilon_i x_i^3
    + {(\hat \beta - \beta)}^2 n^{-1} \sum_{i=1}^n x_i^4\]

    Now, consider the convergence properties of these terms in turn. Since $\{y_i, x_i\}$ is jointly stationary and ergodic, $\{\varepsilon_i^2 x_i^2\}$ is
    stationary and ergodic. Moreover, the mean exists by assumption. Hence, by Kolmogorov's SLLN:\
    \[n^{-1} \sum_{i=1}^n \varepsilon_i^2 x_i^2 \xrightarrow{a.s.} \mathbb E[\varepsilon_i^2 x_i^2]\]

    Next, consider the term $n^{-1} \sum_{i=1}^n \varepsilon_i x_i^3$. Note that $\{\varepsilon_i x_i^3\}$ is also stationary ergodic. Moreover, by the
    Cauchy-Schwarz inequality:
    \[\mathbb E[|\varepsilon_i x_i^3|] = \mathbb E[|(\varepsilon_i x_i) x_i^2|] \le
    {\left( \mathbb E[\varepsilon_i^2 x_i^2] \mathbb E[x_i^4] \right)}^{\frac 12}\]

    Since the means on the R.H.S. exist and are finite, $\mathbb E[\varepsilon_i x_i^3]$ does as well. Hence, by the SLLN:\
    \[n^{-1} \sum_{i=1}^n \varepsilon_i x_i^3 \xrightarrow{a.s.} \mathbb E[\varepsilon_i x_i^3]\]

    Since $\hat \beta$ is consistent ($\hat \beta - \beta \xrightarrow{p} 0$), the continuous mapping theorem implies:\
    \[2(\hat \beta - \beta) n^{-1} \sum_{i=1}^n \varepsilon_i x_i^3 \xrightarrow{p} 0\]
    
    Finally, consider $n^{-1} \sum_{i=1}^n x_i^4$. Given the ergodic stationarity and the fourth moment assumption, the SLLN implies:\
    \[n^{-1} \sum_{i=1}^n x_i^4 \xrightarrow{a.s.} \mathbb E[x_i^4]\]

    Using the same argument as above, we have:\
    \[{(\hat \beta - \beta)}^2 n^{-1} \sum_{i=1}^n x_i^4 \xrightarrow{p} 0\]

    Combining these convergence results, we have:
    \[n^{-1} \sum_{i=1}^n \hat \varepsilon_i^2 x_i^2 \xrightarrow{p} \mathbb E[\varepsilon_i^2 x_i^2]\]

    Note that the convergence is strong if the consistency of $\hat \beta$ is strong.
\end{proof}

\newpage

\textbf{8. (Population analogue of deviation from the mean regression formula.)}
\claim{
    Let $(y, x)$ be a random variable where $y$ is $\mathbb R$-valued and $x$ is $\mathbb R^K$-valued, with the first term of $x$ being one. Let
    $x = (1, \tilde x)$, so that $\tilde x$ contains the nonconstant covariates. Denote, by $\hat {\mathbb E}^*(y|x)$, the linear projection of $y$ on $x$,
    assuming $\mathbb E[xx']$ is nonsingular. Then:\
    \[\hat {\mathbb E}^*(y|x) = \mu + \gamma'\tilde x\]
    where:
    \begin{align*}
        & \gamma = \mathbb V{(\tilde x)}^{-1} \mathbb C(\tilde x, y) \\
        & \mu = \mathbb E(y) - \mathbb E(\tilde x)'\gamma
    \end{align*}
}

\begin{proof}
    \begin{align*}
        \hat {\mathbb E}^*(y|x) &= x' {(\mathbb E[xx'])}^{-1} \mathbb E[xy] \\
        &= \begin{bmatrix}
            1 & \tilde x'
        \end{bmatrix}
        {\left( \mathbb E
        \begin{bmatrix}
            1 & \tilde x' \\
            \tilde x & \tilde x \tilde x'
        \end{bmatrix}
        \right)}^{-1}
        \mathbb E
        \begin{bmatrix}
            y \\
            \tilde x y
        \end{bmatrix} \\
        &= \begin{bmatrix}
            1 & \tilde x'
        \end{bmatrix}
        \begin{bmatrix}
            1 & \mathbb E[\tilde x'] \\
            \mathbb E[\tilde x] & \mathbb E[\tilde x \tilde x']
        \end{bmatrix}^{-1}
        \begin{bmatrix}
            \mathbb E[y] \\
            \mathbb E[\tilde x y]
        \end{bmatrix} \\
        &= \begin{bmatrix}
            1 & \tilde x'
        \end{bmatrix}
        \begin{bmatrix}
            1 + \mathbb E[\tilde x'] {\left( \mathbb E[\tilde x \tilde x'] - \mathbb E[\tilde x] \mathbb E[\tilde x']\right)}^{-1}\mathbb E[\tilde x] & -\mathbb E[\tilde x'] {\left( \mathbb E[\tilde x \tilde x'] - \mathbb E[\tilde x] \mathbb E[\tilde x']\right)}^{-1} \\
            -{\left( \mathbb E[\tilde x \tilde x'] - \mathbb E[\tilde x] \mathbb E[\tilde x']\right)}^{-1} \mathbb E[\tilde x] & {\left( \mathbb E[\tilde x \tilde x'] - \mathbb E[\tilde x] \mathbb E[\tilde x']\right)}^{-1}
        \end{bmatrix}
        \begin{bmatrix}
            \mathbb E[y] \\
            \mathbb E[\tilde x y]
        \end{bmatrix} \\
        &= \begin{bmatrix}
            1 & \tilde x'
        \end{bmatrix}
        \begin{bmatrix}
            1 + \mathbb E[\tilde x']\mathbb V{[\tilde x]}^{-1} \mathbb E[\tilde x] & -\mathbb E[\tilde x']\mathbb V{[\tilde x]}^{-1} \\
            -\mathbb V{[\tilde x]}^{-1} \mathbb E[\tilde x] & \mathbb V{[\tilde x]}^{-1}
        \end{bmatrix}
        \begin{bmatrix}
            \mathbb E[y] \\
            \mathbb E[\tilde x y]
        \end{bmatrix} \\
        &= \begin{bmatrix}
            1 + \mathbb E[\tilde x']\mathbb V{[\tilde x]}^{-1} \mathbb E[\tilde x] - \tilde x' \mathbb V{[\tilde x]}^{-1} \mathbb E[\tilde x] & -\mathbb E[\tilde x']\mathbb V{[\tilde x]}^{-1} + \tilde x' \mathbb V{[\tilde x]}^{-1} \\
        \end{bmatrix}
        \begin{bmatrix}
            \mathbb E[y] \\
            \mathbb E[\tilde x y]
        \end{bmatrix} \\
        &= \mathbb E[y] + \mathbb E[\tilde x']\mathbb V{[\tilde x]}^{-1} \mathbb E[\tilde x] \mathbb E[y] - \tilde x' \mathbb V{[\tilde x]}^{-1} \mathbb E[\tilde x] \mathbb E[y] - \mathbb E[\tilde x']\mathbb V{[\tilde x]}^{-1} \mathbb E[\tilde x y] + \tilde x' \mathbb V{[\tilde x]}^{-1} \mathbb E[\tilde x y] \\
        &= \mathbb E[y] + \mathbb E[\tilde x']{\left( \mathbb V{[\tilde x]}^{-1} \mathbb E[\tilde x] \mathbb E[y] - \mathbb V{[\tilde x]}^{-1} \mathbb E[\tilde x y]\right)} + \tilde x'{\left( \mathbb V{[\tilde x]}^{-1} \mathbb E[\tilde x y] - \mathbb V{[\tilde x]}^{-1} \mathbb E[\tilde x] \mathbb E[y]\right)} \\
        &= \mathbb E[y] - \mathbb E[\tilde x']\mathbb V{[\tilde x']}^{-1} \mathbb C[\tilde x, y] + \tilde x' \mathbb V{[\tilde x]}^{-1} \mathbb C[\tilde x, y] \\
        &= \mu + \tilde x' \gamma
    \end{align*}
\end{proof}

\newpage

\textbf{11. (Breusch-Godfrey test for serial correlation)}
\claim{
    Consider the standard regression framework. Let $X$ denote the data matrix, and let:
    \begin{align*}
        & E = \begin{bmatrix}
            0 & 0 & \cdots & 0 \\
            \hat \varepsilon_1 & 0 & \cdots & 0 \\
            \hat \varepsilon_2 & \hat \varepsilon_1 & \cdots & 0 \\
            \hat \varepsilon_3 & \hat \varepsilon_2 & \cdots & 0 \\
            \vdots & \vdots & \ddots & \vdots \\
            \hat \varepsilon_{n-1} & \hat \varepsilon_{n-2} & \cdots & \hat \varepsilon_{n-p}
        \end{bmatrix} &&
        \hat B = \begin{bmatrix}
            n^{-1} X'X & n^{-1} X'E \\
            n^{-1} E'X & n^{-1} E'E
        \end{bmatrix} &&
        \hat B^{-1} =
            \begin{bmatrix}
                \hat B^{11} & \hat B^{12} \\
                \hat B^{21} & \hat B^{22}
            \end{bmatrix} \\
        & \hat \varepsilon = \begin{bmatrix}
            \hat \varepsilon_1 & \cdots & \hat \varepsilon_n
        \end{bmatrix}' && \hat \gamma = \begin{bmatrix}
            \hat \gamma_1 & \cdots & \hat \gamma_p
        \end{bmatrix}'
    \end{align*}
    Then:
    \begin{enumerate}[(a)]
        \item In the auxiliary regression of $\hat \varepsilon_t$ on $(x_t, \hat \varepsilon_{t-1}, \ldots, \hat \varepsilon_{t-p})$:
        \[\hat \alpha = \hat B^{-1} \begin{bmatrix}
            0 \\ \hat \gamma
        \end{bmatrix}\]
        where $\hat \alpha$ denotes the OLS estimator in the auxiliary regression.
        \item $\hat B \xrightarrow{p} B$, where:
        \begin{align*}
            & B = \begin{bmatrix}
                \mathbb E(x_t x_t') & H \\
                H' & \sigma^2 I
            \end{bmatrix} \\
            & H = \begin{bmatrix}
                \mathbb E(x_t \varepsilon_{t-1}) & \cdots & \mathbb E(x_t \varepsilon_{t-p})
            \end{bmatrix}
        \end{align*}
        \item $\hat \alpha \xrightarrow{p} 0$.
        \item \[\frac{\text{SSR}}{n-K-p} \xrightarrow{p} \sigma^2\]
        where $\text{SSR}$ denote the sum of squared residuals from the auxiliary regression.
        \item \[pF = \frac{n \hat \gamma' \hat B^{22} \hat \gamma}{\text{SSR}/(n - K - p)}\]
        where $F$ denotes the $F$ statistic corresponding with the hypothesis that the coefficients on the lagged residuals are equal to zero.
        \item \[\hat B^{22} = {\left[ n^{-1} E'E - {\left( n^{-1} E'X \right)} n^{-1} X'X  {\left( n^{-1} X'E \right)}\right]}^{-1}\]
        \item $Q_{BP}^* - pF \xrightarrow{p} 0$, where:
        \[Q_{BP}^* = n \hat \rho'{(I - \hat \Phi)}^{-1} \hat \rho\]
        is the modified Box-Pierce statistic used for testing whether the error term is autocorrelated, $\hat \rho$ denotes the vector of $p$ sample
        autocorrelations of the residuals, and $\hat \Phi$ is defined elementwise by:
        \[\hat \phi_{jk} = n^{-1} \sum_{t=j+1}^n x_t \hat \varepsilon_{t-j} {\left( n^{-1} \sum_{t=1}^n x_t x_t' \right)}^{-1}
        n^{-1} \sum_{t=k+1}^n x_t \hat \varepsilon_{t-k} {\left( {[n - K]}^{-1} \sum_{t=1}^n \hat \varepsilon_t^2 \right)}^{-1}\]
    \end{enumerate}
}

\begin{proof}
    (a): The data matrix corresponding to the auxiliary regression is: $\Xi = \begin{bmatrix}
        X & E
    \end{bmatrix}$. So:
    \begin{align*}
        \hat \alpha &= {(\Xi'\Xi)}^{-1} \Xi' \hat \varepsilon \\
        &= {(n^{-1}\Xi'\Xi)}^{-1} n^{-1}\Xi' \hat \varepsilon \\
        &= {\left( n^{-1}\begin{bmatrix}
        X' \\ E'
        \end{bmatrix}
        \begin{bmatrix}
            X & E
        \end{bmatrix}
        \right)}^{-1} n^{-1}
        \begin{bmatrix}
            X' \\ E'
        \end{bmatrix}
        \hat \varepsilon \\
        &= \begin{bmatrix}
            n^{-1} X'X & n^{-1} X'E \\
            n^{-1} E'X & n^{-1} X'X
        \end{bmatrix}^{-1}
        \begin{bmatrix}
            n^{-1} X'\hat \varepsilon \\
            n^{-1} E'\hat \varepsilon
        \end{bmatrix} \\
        &= \hat B^{-1} \begin{bmatrix}
            n^{-1} X'\hat \varepsilon \\
            n^{-1} E'\hat \varepsilon
        \end{bmatrix}
    \end{align*}

    Now, notice:
    \begin{align*}
        X'\hat \varepsilon &= X'(y - X \hat \beta) = X'y - X'X \hat \beta \\
        &= X'y - X'y = 0
    \end{align*}

    And:
    \[E'\hat \varepsilon  = n^{-1} \begin{bmatrix}
        n^{-1} \displaystyle \sum_{t=1}^n \hat \varepsilon_t \hat \varepsilon_{t-1} \\
        \vdots \\
        n^{-1} \displaystyle \sum_{t=1}^n \hat \varepsilon_t \hat \varepsilon_{t-p}
    \end{bmatrix}
    = \begin{bmatrix}
        \hat \gamma_1 \\
        \vdots \\
        \hat \gamma_p
    \end{bmatrix}
    = \hat \gamma
    \]

    So:
    \[\hat \alpha = \hat B^{-1} \begin{bmatrix}
        0 \\ \hat \gamma
    \end{bmatrix}\]

    (b): Note that convergence in probability of a partitioned matrix is equivalent to convergence in probability of the individual components. Hence, to
    show that $\hat B \xrightarrow{p} B$, we can show that the components of $\hat B$ converge in probability to the corresponding components of $B$.

    \medbreak{}

    First, notice that $n^{-1} X'X = n^{-1} \sum_{t=1}^n x_t x_t'$. Since $\{(y_t, x_t)\}$ is stationary ergodic, $\{x_t x_t'\}$ is stationary ergodic. Moreover,
    the mean exists, so by the WLLN for stationary ergodic processes:
    \[n^{-1} X'X = n^{-1} \sum_{t=1}^n x_t x_t' \xrightarrow{p} \mathbb E[x_t x_t']\]

    Next, notice that:
    \begin{align*}
        n^{-1} X'E &= n^{-1} \begin{bmatrix}
            x_1 & \cdots & x_n
        \end{bmatrix}
        \begin{bmatrix}
            0 & 0 & \cdots & 0 \\
            \hat \varepsilon_1 & 0 & \cdots & 0 \\
            \hat \varepsilon_2 & \hat \varepsilon_1 & \cdots & 0 \\
            \hat \varepsilon_3 & \hat \varepsilon_2 & \cdots & 0 \\
            \vdots & \vdots & \ddots & \vdots \\
            \hat \varepsilon_{n-1} & \hat \varepsilon_{n-2} & \cdots & \hat \varepsilon_{n-p}
        \end{bmatrix} \\
        &= \begin{bmatrix}
            n^{-1} \displaystyle \sum_{t=1}^n x_t \hat \varepsilon_{t-1} & \cdots & n^{-1} \displaystyle \sum_{t=1}^n x_t \hat \varepsilon_{t-p}
        \end{bmatrix}
    \end{align*}

    Now, for $k \in [p]$:
    \begin{align*}
        x_t \hat \varepsilon_{t-k} &= x_t (y_{t-k} - x_{t-k}' \hat \beta) \\
        &= x_t(x_{t-k}' \beta + \varepsilon_{t-k} - x_{t-k}'\hat \beta) \\
        &= x_t \varepsilon_{t-k} - x_t x_{t-k}'(\hat \beta - \beta)
    \end{align*}

    So:
    \[n^{-1} \sum_{t=1}^n x_t \hat \varepsilon_{t-k} = n^{-1} \sum_{t=1}^n x_t \varepsilon_{t-k} - {\left( n^{-1} \sum_{t=1}^n x_t x_{t-k}' \right)}(\hat \beta - \beta)\]

    Again, given ergodic stationarity and assuming the relevant moments exist:
    \begin{align*}
        x_t \hat \varepsilon_{t-k} \xrightarrow{p} \mathbb E[x_t \varepsilon_{t-k}] - \mathbb E[x_t x_{t-k}'] \cdot 0 = \mathbb E[x_t \varepsilon_{t-k}]
    \end{align*}

    Hence:
    \[n^{-1} X'E \xrightarrow{p} \begin{bmatrix}
        \mathbb E[x_t \varepsilon_{t-1}] & \cdots & \mathbb E[x_t \varepsilon_{t-p}]
    \end{bmatrix}\]

    \hrulefill{}

    Finally, notice that:
    \begin{align*}
        n^{-1} E'E &= 
        n^{-1} \begin{bmatrix}
            0 & \hat \varepsilon_1 & \hat \varepsilon_2 & \hat \varepsilon_3 & \cdots & \hat \varepsilon_{n-1} \\
            0 & 0 & \hat \varepsilon_1 & \hat \varepsilon_2 & \cdots & \hat \varepsilon_{n-2} \\
            \vdots & \vdots & \vdots & \vdots & \ddots & \vdots \\
            0 & 0 & 0 & 0 & \cdots & \hat \varepsilon_{n-p}
        \end{bmatrix}
        \begin{bmatrix}
            0 & 0 & \cdots & 0 \\
            \hat \varepsilon_1 & 0 & \cdots & 0 \\
            \hat \varepsilon_2 & \hat \varepsilon_1 & \cdots & 0 \\
            \hat \varepsilon_3 & \hat \varepsilon_2 & \cdots & 0 \\
            \vdots & \vdots & \ddots & \vdots \\
            \hat \varepsilon_{n-1} & \hat \varepsilon_{n-2} & \cdots & \hat \varepsilon_{n-p}
        \end{bmatrix} \\
        &= \begin{bmatrix}
            n^{-1} \displaystyle \sum_{t=1}^n \hat \varepsilon_{t-1}^2 & \cdots & n^{-1} \displaystyle \sum_{t=1}^n \hat \varepsilon_{t-1} \hat \varepsilon_{t-p} \\
            \vdots & \ddots & \vdots \\
            n^{-1} \displaystyle \sum_{t=1}^n \hat \varepsilon_{t-p} \hat \varepsilon_{t-1} & \cdots & n^{-1} \displaystyle \sum_{t=1}^n \hat \varepsilon_{t-p}^2
        \end{bmatrix}
    \end{align*}

    Now, for $j, k \in [n]$:
    \[\hat \varepsilon_{t-j} \hat \varepsilon_{t-k} = x_{{t-j}}'(\beta - \hat \beta) x_{t-k}'(\beta - \hat \beta) + x_{t-j}'(\beta - \hat \beta) \varepsilon_{t-k} +
    \varepsilon_{t-j} x_{t-k}'(\beta - \hat \beta) + \varepsilon_{t-j} \varepsilon_{t-k}\]
    
    Clearly, given $\hat \beta \xrightarrow{p} \beta$, we have:
    \[n^{-1} \hat \varepsilon_{t-j} \hat \varepsilon_{t-k} \xrightarrow{p} \mathbb E[\varepsilon_{t-j} \varepsilon_{t-k}] = \begin{cases}
        \sigma^2 & j = k \\
        0 & j \ne k
    \end{cases}\]

    So:
    \[n^{-1} E'E \xrightarrow{p} \sigma^2 I\]

    Thus, we have shown:
    \[\hat B = \begin{bmatrix}
        n^{-1} X'X & n^{-1} X'E \\
        n^{-1} E'X & n^{-1} E'E
    \end{bmatrix}
    \xrightarrow{p}
    \begin{bmatrix}
        \mathbb E[x_t x_t'] & H \\
        H' & \sigma^2 I
    \end{bmatrix} = B
    \]

    (c): Recall, $\hat \alpha = \hat B^{-1} \begin{bmatrix}0 & \hat \gamma\end{bmatrix}'$. Since matrix inversion is continuous, by the continuous mapping theorem
    and the previous result, $\hat B^{-1} \xrightarrow{p} B^{-1}$. And, given that $\hat \gamma$ is consistent for $\gamma = 0$:
    \[\hat \alpha \xrightarrow{p} B^{-1} \begin{bmatrix}
        0 \\ 0
    \end{bmatrix} = 0\]

    (d): Let $\hat u$ denote the vector of residuals in the auxiliary regression. Then:
    \begin{align*}
        \text{SSR} &= \hat u' \hat u \\
        &= (\hat \varepsilon - \Xi \hat \alpha)'(\hat \varepsilon - \Xi \hat \alpha) \\
        &= \hat \varepsilon'\hat \varepsilon - 2 \hat \alpha \Xi' \hat \varepsilon + \hat \alpha \Xi' \Xi \hat \alpha \\
        &= \hat \varepsilon' \hat \varepsilon - \hat \alpha{\left( 2\Xi'\hat \varepsilon - \Xi'\Xi \hat \alpha \right)} \\
        &= \hat \varepsilon'\hat \varepsilon - \hat \alpha{\left( 2 \Xi'\hat \varepsilon- \Xi'\Xi {(\Xi'\Xi)}^{-1}\Xi'\hat \varepsilon \right)} \\
        &= \hat \varepsilon'\hat \varepsilon - \hat \alpha \Xi'\hat \varepsilon \\
        &= \hat \varepsilon'\hat \varepsilon - \hat \alpha \begin{bmatrix}X & E\end{bmatrix}'\hat \varepsilon \\
        &= \hat \varepsilon'\hat \varepsilon - \hat \alpha \begin{bmatrix}
            X' \hat \varepsilon \\
            E' \hat \varepsilon
        \end{bmatrix} \\
        &= \hat \varepsilon' \hat \varepsilon - \hat \alpha \begin{bmatrix}
            0 \\
            n \hat \gamma
        \end{bmatrix} \\
        \implies n^{-1} \text{SSR} &= n^{-1} \sum_{t=1}^n \hat \varepsilon_t^2 - \hat \alpha \begin{bmatrix}
            0 \\
            \hat \gamma
        \end{bmatrix} \\
        & \xrightarrow{p} \mathbb E[\hat \varepsilon_t^2] - 0 \cdot \begin{bmatrix}
            0 \\
            0
        \end{bmatrix} \\
        &= \sigma^2
    \end{align*}

    (e): In matrix form, the hypothesis that the coefficients on the lagged residuals are equal to zero is:
    \[
        \begin{bmatrix}
            0_{p \times K} & I_{p \times p}
        \end{bmatrix}
        \alpha = 0
    \]
    Letting $R = \begin{bmatrix}
        0 & I
    \end{bmatrix}$, the $F$ statistic is defined as:
    \[F = \frac{(R\hat \alpha)' {\left[ R{(\Xi'\Xi)}^{-1}R' \right]}^{-1} R\hat \alpha / p}{\text{SSR}/(n-K-p)}\]

    Notice:
    \[
        R \hat \alpha = \begin{bmatrix}
            0 & I
        \end{bmatrix} \hat B^{-1} \begin{bmatrix}
            0 \\ \hat \gamma
        \end{bmatrix}
        = R \hat \alpha = \begin{bmatrix}
            0 & I
        \end{bmatrix}
        \begin{bmatrix}
            \hat B^{11} & \hat B^{12} \\
            \hat B^{21} & \hat B^{22}
        \end{bmatrix}
        \begin{bmatrix}
            0 \\ \hat \gamma
        \end{bmatrix}
        = \hat B^{12} \hat \gamma
    \]
    And:
    \begin{align*}
        R{(\Xi'\Xi)}^{-1}R' &= \begin{bmatrix}
            0 & I
        \end{bmatrix} {(n \hat B)}^{-1}
        \begin{bmatrix}
            0 \\ I
        \end{bmatrix} \\
        &= n^{-1} \begin{bmatrix}
            0 & I
        \end{bmatrix}
        \begin{bmatrix}
            \hat B^{11} & \hat B^{12} \\
            \hat B^{21} & \hat B^{22}
        \end{bmatrix}
        \begin{bmatrix}
            0 \\ I
        \end{bmatrix} \\
        &= n^{-1} \hat B^{22}
    \end{align*}

    Hence:
    \begin{align*}
        F &= \frac{n \gamma' {(\hat B^{22})}^{-1} \hat B^{22} \hat \gamma / p}{\text{SSR}/(n-K-p)} \\
        \implies pF &= \frac{\hat \gamma' \hat B^{22} \hat \gamma}{\text{SSR}/(n-K-p)}
    \end{align*}

    (f): This follows trivially from the formula for calculating the inverse of a partitioned matrix, which provides that the lower right block of the
    $\hat B^{-1}$, in terms of the components of $\hat B$, is:
    \[{\left( n^{-1} E'E - n^{-1} E'X{(n^{-1} X'X)}^{-1} n^{-1} X'E \right)}^{-1}\]

    (g): Consider the probability limit of $\hat B^{22} = {\left[ n^{-1} E'E - n^{-1} E'X{(n^{-1} X'X)}^{-1} n^{-1} X'E \right]}^{-1}$. As we've shown,
    $n^{-1} E'E \xrightarrow{p} \sigma^2 I$ and $n^{-1} X'X \xrightarrow{p} \mathbb E[x_t x_t']$. Now, notice:
    \[n^{-1} E'X = \begin{bmatrix}
        n^{-1} \displaystyle \sum_{t=1}^n x_t \hat \varepsilon_{t-1} \\
        \vdots \\
        n^{-1} \displaystyle \sum_{t=1}^n x_t \hat \varepsilon_{t-p}
    \end{bmatrix}\]

    For $k \in [p]$, $x_t \hat \varepsilon_{t-k} = x_t x_{t-k}'(\beta - \hat \beta) + x_t \varepsilon_{t-k}$. Assuming $\mathbb E[x_t x_{t-k}']$ is finite for
    all $k \in [p]$, by standard arguments, we have:
    \[n^{-1} \sum_{t=1}^n E'X \xrightarrow{p} \begin{bmatrix}
        \mathbb E[x_t \varepsilon_{t-1}] \\
        \vdots \\
        \mathbb E[x_t \varepsilon_{t-p}]
    \end{bmatrix}\]

    It thus follows, by the continuous mapping theorem, that:
    \[\hat B^{22} \xrightarrow{p} {(I - \Phi)}^{-1}/\sigma^2\]
    where $\Phi$ is defined elementwise by:
    \[\phi_{jk} = \mathbb E[x_t \varepsilon_{t-j}] {\left( \mathbb E[x_t x_t'] \right)}^{-1} \mathbb E[x_t \varepsilon_{t-k}]\]

    Noting that $\text{SSR}/(n-K-p) = s^2 \xrightarrow{p} \sigma^2$ and $\hat \gamma \xrightarrow{p} \gamma$, we have that:
    \[n^{-1} pF = \frac{\hat \gamma' \hat B^{22} \hat \gamma}{\text{SSR}/(n-K-p)}\xrightarrow{p} \hat \gamma' {(I - \Phi)}^{-1} \hat \gamma/\sigma^4\]

    It is trivial to see that this is also the probability limit of $Q_{BP}^*$. Hence, $Q_{BP}^* - pF \xrightarrow{p} 0$.
\end{proof}

\end{document}